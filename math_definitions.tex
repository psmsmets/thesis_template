%%%%%%%%%%%%%%%%%%%%%%%%%%%%%%%%%%%%%%%%%%%%%%%%%%%%%%%%%%%%%%
%
% 									MATH OPERATORS
%
%%%%%%%%%%%%%%%%%%%%%%%%%%%%%%%%%%%%%%%%%%%%%%%%%%%%%%%%%%%%%%
 
\newcommand{\signal}{P'}
\newcommand{\amplitude}{A}
\newcommand{\otherpsi}{\breve\psi}
\newcommand{\potential}{\breve\varphi}
\newcommand{\der}{\mathrm{d}}
\newcommand{\Dt}{\mathrm{D}_t}
\newcommand{\Dtbar}{\xbar{\mathrm{D}}_t}
\newcommand{\snr}{\text{SNR}}
\newcommand{\degr}{\ensuremath{^\circ}}
\newcommand{\dwdx}{\dfrac{\partial \mathbf{w}}{\partial \mathbf{x}}}
\newcommand{\dcdx}{\dfrac{\partial c_T}{\partial \mathbf{x}}}
\newcommand{\dwdr}{\dfrac{\partial \mathbf{w}_\mathbf{r}}{\partial \mathbf{r}}}
\newcommand{\dcdr}{\dfrac{\partial c_T}{\partial \mathbf{r}}}
\providecommand{\xbar}[1]{\bar{#1}}
\providecommand{\abs}[1]{\lvert#1\rvert}
\providecommand{\norm}[1]{\lVert#1\rVert}
\providecommand{\variance}[1]{\langle#1\rangle}
\providecommand{\expectation}[1]{\langle#1\rangle}
%\providecommand{\expectation}[1]{\text{E}\!\left[#1\right]}
\providecommand{\diag}[1]{\operatorname{diag}\!\left(#1\right)}
\providecommand{\diagentry}[1]{\makebox[3em]{$#1$}}

% Here is a \perp, upside down, called here \tang, and it obeys math styles.
% https://tex.stackexchange.com/questions/331094/tangential-part
\def\tang{\ThisStyle{\abovebaseline[0pt]{\scalebox{-1}{$\SavedStyle\perp$}}}}