\begin{tikzpicture}[xscale=0.0325,yscale=.0525]

\tikzstyle{ax} = [font=\small, black];
\tikzstyle{tick} = [font=\footnotesize, black];
\tikzstyle{pause} = [fill=lighter-grey];


%% MATLAB code:
% GPHeight=0:5:120;
% [T pres zwind] = atmoscira( 50, 'GPHeight', GPHeight*1000, 'Monthly', 1 );
% [T pres zwind] = atmoscira( 50, 'GPHeight', GPHeight*1000, 'Monthly', 7 );

%% Temperature:
% fprintf('plot [smooth] coordinates {');
% for i=1:length(GPHeight),
%	fprintf('(%.1f,%d) ',T(i),GPHeight(i));
% end;
% fprintf('}\n');

%% Wind:
% fprintf('plot [smooth] coordinates {');
% for i=1:length(GPHeight),
%	fprintf('(%.1f,%d) ',zwind(i),GPHeight(i));
% end;
% fprintf('}\n');


%% The magical curves :

\newcommand{\atmosciraTempJan}%
{ %
	plot [smooth] coordinates {(4.6,0) (-28.6,5) (-54.3,10) (-55.2,15) (-56.3,20) (-54.9,25) (-46.9,30) (-37.7,35) (-26.7,40) (-18.7,45) (-19.6,50) (-31.4,55) (-42.4,60) (-46.9,65) (-48.1,70) (-49.3,75) (-51.7,80) (-59.4,85) (-71.8,90) (-80.9,95) (-79.9,100) (-68.7,105) (-40.1,110) (20.8,115) (96.2,120) } %
}

\newcommand{\atmosciraTempJul}%
{ %
	plot [smooth] coordinates {(12.2,0) (-9.5,5) (-43.1,10) (-53.3,15) (-51.5,20) (-47.3,25) (-37.4,30) (-24.7,35) (-11.6,40) (-2.2,45) (-2.0,50) (-10.9,55) (-24.8,60) (-42.5,65) (-59.2,70) (-77.0,75) (-93.5,80) (-104.8,85) (-109.9,90) (-107.2,95) (-86.2,100) (-49.7,105) (-4.4,110) (44.7,115) (115.2,120) } %
}

\newcommand{\atmosciraZWindJan}%
{ %
	plot [smooth] coordinates {(2.5,0) (10.3,5) (16.0,10) (20.6,15) (19.0,20) (21.9,25) (26.7,30) (30.1,35) (33.8,40) (39.3,45) (43.7,50) (43.3,55) (39.8,60) (36.0,65) (31.8,70) (27.6,75) (22.6,80) (13.4,85) (1.0,90) (-9.1,95) (-16.4,100) (-25.3,105) (-36.1,110) (-40.2,115) (-37.9,120) } %
}

\newcommand{\atmosciraZWindJul}%
{ %
	plot [smooth] coordinates {(1.9,0) (7.6,5) (13.6,10) (8.3,15) (-2.1,20) (-8.8,25) (-13.3,30) (-19.4,35) (-26.9,40) (-34.3,45) (-40.6,50) (-47.8,55) (-56.2,60) (-64.8,65) (-70.5,70) (-64.8,75) (-43.6,80) (-11.3,85) (16.7,90) (33.6,95) (39.4,100) (32.8,105) (20.7,110) (13.3,115) (11.4,120) } %
}


\pgfmathsetmacro{\xmin}{-115}
\pgfmathsetmacro{\xmax}{40}
\pgfmathsetmacro{\ymin}{0}
\pgfmathsetmacro{\ymax}{130}

% tropopause
\fill[pause] (\xmin,11) rectangle (\xmax,13);
% stratopause
\fill[pause] (\xmin,45.5) rectangle (\xmax,48.5);
% mesopause
\fill[pause] (\xmin,95) rectangle (\xmax,98);

%\draw[ax] (\xmin,\ymin) rectangle (\xmax,\ymax);
\draw[ax] (\xmin,\ymin) -- (\xmax,\ymin);
\draw[ax] (\xmin,\ymin) -- (\xmin,\ymax);
%\draw[ax] (\xmax,\ymin) -- (\xmax,\ymax);
%\draw[ax] (\xmin,\ymax) -- (\xmax,\ymax);

\begin{scope} % surface tension
	\path [clip] (\xmin,\ymin) rectangle (\xmax,\ymax);
	\draw[black,very thick, dashed] \atmosciraTempJul;
	\draw[black,very thick] \atmosciraTempJan;
\end{scope}

% axis ticks (temperature)
\foreach \x in {-100,-75,...,25}{
	\draw [black] (\x,\ymin-2) -- (\x,\ymin) node[pos=0.,below,anchor=north,tick]{\x};
}
\node[ax] at (-37.5,-15) {Temperature($\degr$C)};

% axis ticks (altitude)
\foreach \y in {0,20,...,120}{
	\draw [black] (\xmin-3,\y) -- (\xmin,\y) node[pos=0.,below,anchor=east,tick]{\y};
%	\draw [black] (\xmax,\y) -- (\xmax+2,\y);
}
\node[ax,rotate=90] at (-145,60) {Geopotential height(km)};


%%%%%%%%%%%%%%%
%%.          ZWind             %%
%%%%%%%%%%%%%%%

\begin{scope}[shift={(170,0)}]

\pgfmathsetmacro{\xmin}{-90}
\pgfmathsetmacro{\xmax}{65}
\pgfmathsetmacro{\ymin}{0}
\pgfmathsetmacro{\ymax}{130}

% tropopause
\fill[pause] (\xmin,11) rectangle (\xmax,13);
% stratopause
\fill[pause] (\xmin,45.5) rectangle (\xmax,48.5);
% mesopause
\fill[pause] (\xmin,95) rectangle (\xmax,98);

%\draw[ax] (\xmin,\ymin) rectangle (\xmax,\ymax);
\draw[ax] (\xmin,\ymin) -- (\xmax,\ymin);
%\draw[ax] (\xmin,\ymin) -- (\xmin,\ymax);
%\draw[ax] (\xmax,\ymin) -- (\xmax,\ymax);
%\draw[ax] (\xmin,\ymax) -- (\xmax,\ymax);

\begin{scope} % surface tension
	\path [clip] (\xmin,\ymin) rectangle (\xmax,\ymax);
	\draw[very thick] \atmosciraZWindJan;
	\draw[very thick, dashed] \atmosciraZWindJul;
\end{scope}

% axis ticks (zonal wind)
\foreach \x in {-75,-50,...,50}{
	\draw [black] (\x,\ymin-2) -- (\x,\ymin) node[pos=0.,below,anchor=north,tick]{\x};
}
\node[ax] at (-12.5,-15) {Zonal wind($\text{m}\,\text{s}^{-1}$)};

%% labels

\pgfmathsetmacro{\xc}{-110}

% labels
\node[tick,grey,fill=white] at (\xc,12) {Tropopause};
\node[tick,grey,fill=white] at (\xc,47) {Stratopause};
\node[tick,grey,fill=white] at (\xc,96.5) {Mesopause};

\node[ax] at (\xc,5) {Troposphere};
\node[ax] at (\xc,30) {Stratosphere};
\node[ax] at (\xc,70) {Mesosphere};
\node[ax] at (\xc,125) {Thermosphere};

\end{scope}
%
\end{tikzpicture}